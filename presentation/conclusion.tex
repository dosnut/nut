
\section{Zusammenfassung}
\subsection{Konkurenz}
\begin{frame}[<+-| alert@+>]{Andere Netzwerkmanager}
	\fontsize{7}{8.4}\selectfont
	\begin{center}
	\begin{tabular}{cc|c|c|c|c|c}
		&					& kwifi		& gtkwifi	& wifi radar	& NetworkManager	& NUt \\
		\hline
		&Environments		& $X$			& $X$			& $X$			& $X$				& $\surd$ \\
		\hline
		\multirow{3}{*}{\begin{sideways} Gerät \end{sideways}}
		& Ethernet			& $X$			& $X$			& $X$			& $\surd$			& $\surd$ \\
		\cline{2-7}
		& Drahtlos			& $\surd$		& $\surd$		& $\surd$		& $\surd$			& $\surd$ \\
		\cline{2-7}
		& PPP				& $X$			& $X$			& $X$			& $\surd$			& $X^{4}$ \\
		\hline
		\multirow{3}{*}{\begin{sideways} IP-Konf. \end{sideways}}
		&static				& $\surd$		& $\surd$		& $\surd$		& $\surd$			& $\surd$ \\
		\cline{2-7}
		& dhcp				& $\surd$		& $\surd$		& $\surd$		& $\surd$			& $\surd$ \\
		\cline{2-7}
		& zeroconf			& $X$			& $X$			& $X$			& $\surd$			& $\surd$ \\
		\hline
		\multirow{4}{*}{\begin{sideways} WLAN \end{sideways}}
		&unverschlüsselt	& $\surd$		& $\surd$		& $\surd$		& $\surd$			& $\surd^{3}$ \\
		\cline{2-7}
		&WEP				& $\surd$		& $\surd$		& $\surd$		& $\surd$			& $\surd^{3}$ \\
		\cline{2-7}
		&WPA				& $\surd$		& $\surd$		& $\surd$		& $\surd$			& $\surd^{3}$ \\
		\cline{2-7}
		&Konfig. speichern	& $\surd^{5}$	& $\surd^{5}$	& $\surd^{5}$	& $\surd^{5}$		& $\surd^{3}$
	\end{tabular}
	\end{center}
	\textbf{1} built-in\\
	\textbf{2} external\\
	\textbf{3} über wpa\_supplicant\\
	\textbf{4} mit Skripten möglich aber nicht als Konfigurationsoption verfügbar\\
	\textbf{5} eigene \\
\end{frame}

\begin{frame}[<+-| alert@+>]{Motivation-NetworkManager<=>NUt}
	\begin{tabular}{l|c|c}
								& NetworkManager	& NUt \\
		\hline
		Environments			& $X$					& $\surd$ \\
		\hline
		Server-Skripte			& $X$					& $\surd$ \\
		\hline
		Nutzer-Skripte			& $X$					& $\surd$ \\
		\hline
		QT-GUI					& $\surd$(KDE3)			& $\surd$(Qt4) \\
		\hline
		GTK-GUI					& $\surd$(Gnome)		& $X$			\\
		\hline
		DBus-Interface			& $\surd$				& $\surd$	\\
		\hline
		Client-Library			& $X$					& $\surd$(Qt4)	\\
		\hline
		Autom. Netzwechsel		& $\surd$				& nur pro Gerät \\
		\hline
		Plugin-Struktur			& $\surd$				& $X$
	\end{tabular}
\end{frame}


\subsection{Zukünftiges}
\begin{frame}[<+-|alert@+>]{Mögliche zukünftige Änderungen}
	\begin{itemize}
		\item Bugfixes (sofern weitere nötig)
		\item Server
		\begin{itemize}
			\item Fallback-Konfiguration für DHCP
			\item Signal vom Server falls ein neues Netzwek gefunden wurde
			\item PPP als Konfigurationsoption (momentan nur über Skripte)
		\end{itemize}
		\item libnut/qnut
		\begin{itemize}
			\item Etwas mehr Unabhängigkeit vom wpa\_supplicant
			\item Mehr WLAN-Informationen (z.B. Übertragungsgeschwindigkeit)
		\end{itemize}
		\item Sehr zukünftiges
		\begin{itemize}
			\item DBus Interface generischer machen
			\item Konsolenclient
			\item IPv6
		\end{itemize}
	\end{itemize}
\end{frame}

\subsection{Fazit}
\begin{frame}[<+-|alert@+>]{Warum brauche ich NUt?}
	\begin{itemize}
		\item Es ist kostenlos (GPL)
		\item Es ist schnell
		\item Es verbraucht wenig Speicher und keine CPU-Zeit (im Idle)
		\item Es steuert alle gängigen vom wpa\_supplicant unterstützten
		\item Es steuert alle Netzwerkadapter an
		\item Es ist durch die Skripte enorm flexibel
	\end{itemize}
\end{frame}

\begin{frame}[<+-|alert@+>]{Ende}
	\begin{center}
		\begin{Large}
			Homepage, Debuild Skripte und Ebuild:
		\end{Large}
		\begin{Huge}
			http://repo.or.cz/w/nut.git
		\end{Huge}
	\end{center}
\end{frame}
