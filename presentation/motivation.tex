\section{Motivation}
\subsection{}
\begin{frame}[<+-| alert@+>]{Motivation}
	\begin{itemize}
		\item Konfiguration für ein Laptop in verschiedenen Arbeitsbereichen unpraktisch. \\
		\item Oft durch selbst zusammengehackte Skripte erleichtert, die meist aber mit ``sudo'' ausgeführt werden müssen.
		\item Umständliches Starten aller notwendigen Teile (WLAN Konfigurieren, IP zuweisen, VPN aufbauen)
		\item Unflexibel: Je nach Umgebung andere Konfiguration nötig
	\end{itemize}
\end{frame}

\begin{frame}[fragile]{Motivation-Beispielskript}
\fontsize{4.8}{5.8} \selectfont
\begin{verbatim}
#!/bin/bash
echo "* loading modules...";
echo "  -> acer_acpi";
modprobe -r acer_acpi;
modprobe acer_acpi;
if [ "$1" = "bcm" ]
then
    echo "  -> bcm43xx";
    modprobe bcm43xx;
else
    echo "  -> ndiswrapper";
    modprobe ndiswrapper;
fi
echo "* enabling wlan hardware...";
echo "enabled: 1" > /proc/acpi/acer/wireless;
sleep 0.1;
if [ "$1" = "bcm" ]
then
    iwconfig wlan0 rate 11M;
fi
echo "* bringing up wlan interface...";
if [ "$2" = "connect" -o "$1" = "connect" ]
then
    ifup wlan0;
elif [ "$2" = "connect2" -o "$1" = "connect2" ]
then
    echo "* starting wpa supplicant...";
    wpa_supplicant -B -i wlan0 -D wext -c /etc/wpa_supplicant/wpa_supplicant.conf;
    sleep 1;
    echo "* starting dhcp client"
    dhclient wlan0
fi
\end{verbatim}
\end{frame}

\begin{frame}[<+-| alert@+>]{Motivation-Andere Netzwerkmanager}
	\fontsize{7}{8.4}\selectfont
	\begin{center}
	\begin{tabular}{cc|c|c|c|c|c}
		&					& kwifi		& gtkwifi	& wifi radar	& NetworkManager	& NUt \\
		\hline
		&Environments		& $X$			& $X$			& $X$			& $X$				& $\surd$ \\
		\hline
		\multirow{3}{*}{\begin{sideways} Gerät \end{sideways}}
		& Ethernet			& $X$			& $X$			& $X$			& $\surd$			& $\surd$ \\
		\cline{2-7}
		& Drahtlos			& $\surd$		& $\surd$		& $\surd$		& $\surd$			& $\surd$ \\
		\cline{2-7}
		& PPP				& $X$			& $X$			& $X$			& $\surd$			& $X^{4}$ \\
		\hline
		\multirow{3}{*}{\begin{sideways} IP-Konf. \end{sideways}}
		&static				& $\surd$		& $\surd$		& $\surd$		& $\surd$			& $\surd$ \\
		\cline{2-7}
		& dhcp				& $\surd$		& $\surd$		& $\surd$		& $\surd$			& $\surd$ \\
		\cline{2-7}
		& zeroconf			& $X$			& $X$			& $X$			& $\surd$			& $\surd$ \\
		\hline
		\multirow{4}{*}{\begin{sideways} WLAN \end{sideways}}
		&unverschlüsselt	& $\surd$		& $\surd$		& $\surd$		& $\surd$			& $\surd^{3}$ \\
		\cline{2-7}
		&WEP				& $\surd$		& $\surd$		& $\surd$		& $\surd$			& $\surd^{3}$ \\
		\cline{2-7}
		&WPA				& $\surd$		& $\surd$		& $\surd$		& $\surd$			& $\surd^{3}$ \\
		\cline{2-7}
		&Konfig. speichern	& $\surd^{5}$	& $\surd^{5}$	& $\surd^{5}$	& $\surd^{5}$		& $\surd^{3}$
	\end{tabular}
	\end{center}
	\textbf{1} built-in\\
	\textbf{2} external\\
	\textbf{3} über wpa\_supplicant\\
	\textbf{4} mit Skripten möglich aber nicht als Konfigurationsoption verfügbar\\
	\textbf{5} eigene \\
\end{frame}

\begin{frame}[<+-| alert@+>]{Motivation-NetworkManager<=>NUt}
	\begin{tabular}{l|c|c}
								& NetworkManager	& NUt \\
		\hline
		Environments			& X					& $\surd$ \\
		\hline
		Server-Skripte			& X					& $\surd$ \\
		\hline
		Nutzer-Skripte			& X					& $\surd$ \\
		\hline
		QT-GUI					& $\surd$(KDE3)		& $\surd$(Qt4) \\
		\hline
		GTK-GUI					& $\surd$(Gnome)	& X			\\
		\hline
		DBus-Interface			& $\surd$			& $\surd$	\\
		\hline
		Client-Library			& X					& $\surd$(Qt4)	\\
		\hline
		Autom. Netzwechsel		& $\surd$			& nur pro Gerät
	\end{tabular}

\end{frame}
