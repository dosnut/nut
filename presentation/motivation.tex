\section{Motivation}
\subsection{}
\begin{frame}[<+-| alert@+>]{Motivation}
	\begin{itemize}
		\item Konfiguration für ein Laptop in verschiedenen Arbeitsbereichen unpraktisch. \\
		\item Oft durch selbst zusammengehackte Skripte erleichtert, die meist aber mit ``sudo'' ausgeführt werden müssen.
		\item Umständliches Starten aller notwendigen Teile (WLAN Konfigurieren, IP zuweisen, VPN aufbauen)
		\item Unflexibel: Je nach Umgebung andere Konfiguration nötig
	\end{itemize}
\end{frame}

\begin{frame}[fragile]{Motivation-Beispielskript}
\fontsize{4.8}{5.8} \selectfont
\begin{verbatim}
#!/bin/bash
echo "* loading modules...";
echo "  -> acer_acpi";
modprobe -r acer_acpi;
modprobe acer_acpi;
if [ "$1" = "bcm" ]
then
    echo "  -> bcm43xx";
    modprobe bcm43xx;
else
    echo "  -> ndiswrapper";
    modprobe ndiswrapper;
fi
echo "* enabling wlan hardware...";
echo "enabled: 1" > /proc/acpi/acer/wireless;
sleep 0.1;
if [ "$1" = "bcm" ]
then
    iwconfig wlan0 rate 11M;
fi
echo "* bringing up wlan interface...";
if [ "$2" = "connect" -o "$1" = "connect" ]
then
    ifup wlan0;
elif [ "$2" = "connect2" -o "$1" = "connect2" ]
then
    echo "* starting wpa supplicant...";
    wpa_supplicant -B -i wlan0 -D wext -c /etc/wpa_supplicant/wpa_supplicant.conf;
    sleep 1;
    echo "* starting dhcp client"
    dhclient wlan0
fi
\end{verbatim}
\end{frame}

\begin{frame}[<+-| alert@+>]{Überblick NUt}
	\begin{itemize}
		\item nuts: Daemon(Server) zur Verwaltung der Hardware
		\item libnutclient: Abstraktion der Kommunikation mit dem Server
		\item libnutwireless: Abstraktion der Kommunikation mit dem wpa\_supplicant
		\item QNut: Grafische Benutzeroberfläche in Qt
		\item Genutzte Frameworks/Libraries: Qt4, dbus, libnl, iwlib
	\end{itemize}
\end{frame}
