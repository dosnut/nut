
\section{Server}
\subsection{Aufbau}
\begin{frame}[<+-|alert@+>]{Überblick der Strukturen}
	\includegraphics<+->{nuts_structure.pdf}
	\begin{itemize}
		\item Devices: Entsprechen den Hardwaregeräten
		\begin{itemize}
			\item Erkennt Zustandsänderungen wie Kabel ein-/ausstecken oder WLAN Verbindung.
			\item Kann für WLAN Karten den wpa\_supplicant automatisch starten und beenden.
		\end{itemize}
		\item Environments: Entsprechen Umgebungen; z.Bsp. Arbeitsplatz, Zuhause, ...
		\item Interfaces: Entsprechen je einer IP
	\end{itemize}
\end{frame}

\begin{frame}[<+-|alert@+>]{Environment}
	\begin{itemize}
		\item Environments werden je nach Konfiguration vom Server automatisch ausgewählt, Kriterien für die Auswahl sind:
		\begin{itemize}
			\item Der WLAN Name, in dem sich das Device befindet (ESSID)
			\item Das Vorhandenseins eines Rechners mit einer bestimmten IP (und evtl. passender MAC Adresse)
			\item Wunsch des Benutzers.
		\end{itemize}
	\end{itemize}
\end{frame}

\begin{frame}[<+-|alert@+>]{IP Zuweisung}
	\begin{itemize}
		\item 3 verschiedene Methoden:
		\begin{itemize}
			\item Statisch konfigurierte (oder vom Benutzer eingegebene) IP
			\item Per DHCP (benötigt einen DHCP-Server)
			\item Zeroconf (aka \href{http://tools.ietf.org/html/rfc3927}{``IPv4 Link-Local Adresses'', RFC 3927});
				es wird eine lokal freie IP aus dem Bereich 169.254/16 gesucht. Die IP ist nur im lokalen Netzwerk gültig.
		\end{itemize}
		\item Zur IP gehören weitere Werte, die auch pro Interface konfiguriert werden: Netmask, Gateway, DNS-Server.
	\end{itemize}
\end{frame}

\subsection{Weitere Features}
\begin{frame}[<+-|alert@+>]{Weitere Features}
	\begin{itemize}
		\item Eventgesteurte Scriptausführung: das ermöglich z.Bsp. den Aufbau eines VPN nach dem Aufbau der darunterliegenden Verbindung (die Skripte haben Rootrechte)
		\item Unterstützt Plug'n'Play von Netzwerkgeräten, z.Bsp. PCMCIA Karten oder Laden und Entladen von Treibern wegen Inkompatibilität mit Suspend.
	\end{itemize}
\end{frame}

\subsection{Beispiel}
\begin{frame}[fragile]{Konfigurationsbeispiel}
\textbf{/etc/nuts/nuts.config}
\tiny
\begin{verbatim}
device "eth0" {
    dhcp;
    environment "home" {
        static {
            ip 192.168.0.86;
            netmask 255.255.255.0;
            gateway 192.168.0.1;
            dns-server 192.168.0.1
        };
        select arp 192.168.0.1 00:1A:4F:AA:AA:AA;
    };
    environment "gwlaptop" select arp 129.69.212.17 00:00:1A:19:40:C7;
};

device "eth1" {
    no-auto-start;   // Nicht automatisch aktivieren
    wpa-supplicant config "/etc/wpa_supplicant/wpa_supplicant.conf";
    environment "infeap" select essid "infeap";
    environment "zeroconf" zeroconf;
    environment "user" static user;
};
\end{verbatim}
\end{frame}
