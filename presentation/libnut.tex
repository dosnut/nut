
% TODO
\section{Library}
\subsection{Überblick}
\begin{frame}[<+-| alert@+>]{nut client und wireless client}
	\begin{itemize}
		\item libnutclient : Qt-Library für den Client-Teil der DBus Verbindung zum Server.
		\item Verwaltungsstruktur ähnelt dem Server
		\begin{itemize}
			\item DeviceManager verwaltet die Netzwerkgeräte (Devices)
			\item Device verwaltet die Umgebungen (Environments)
			\item Environment verwaltet die Interfaces
			\item Interface
		\end{itemize}

		\item libnutwireless : Qt-Library für die wpa\_supplicant Verwaltung und Serverkommunikation.
	\end{itemize}
\end{frame}

\subsection{libnutclient}
\begin{frame}[<+-| alert@+>]{nut client - DeviceManager}
	\begin{itemize}
		\item Abstraktion der Serverkommunikation zum DeviceManager
		\item Verwaltung der Devices
		\item Signale zur Kommunikation mit der Applikation
		\begin{itemize}
			\item Server Zustand (an/aus)
			\item Hinzufügen/Entfernen eines Netzwerkgerätes
		\end{itemize}
	\end{itemize}
\end{frame}

\begin{frame}[<+-| alert@+>]{nut client - Device}
	\begin{itemize}
		\item Abstraktion der Serverkommunikation der Netzwerkgeräte
		\item Abstraktion der Kommunikation mit dem wpa\_supplicant und der WirelessExtension
		\item Verwaltung ihrer Environments
		\item Bereitstellung von Netzwerkgeräte-Eigenschaften
		\item Methoden zur Kommunikation mit dem Netzwerkgerät:
		\begin{itemize}
			\item Device an/ausschalten
			\item Setzen eines Environments
		\end{itemize}
		\item Signale zur Kommunikation mit der Applikation
		\begin{itemize}
			\item Statusänderungen des Netzwerkgerätes
			\item Änderung des aktuellen Environments
		\end{itemize}
	\end{itemize}
\end{frame}

\begin{frame}[<+-| alert@+>]{nut client - Environment}
	\begin{itemize}
		\item Abstraktion der Serverkommunikation mit den Environments
		\item Verwaltung ihrer Interfaces
		\item Bereitstellung von Environment-Eigenschaften
		\item Methoden zur Kommunikation mit dem Environment
		\begin{itemize}
			\item Aktivieren des Environments
		\end{itemize}
		\item Signale zur Kommunikation mit der Applikation
		\begin{itemize}
			\item Zustandsänderungen des Environments
		\end{itemize}
	\end{itemize}
\end{frame}

\begin{frame}[<+-| alert@+>]{nut client - Interface}
	\begin{itemize}
		\item Abstraktion der Serverkommunikation mit den Interfaces
		\item Bereitstellung von Informationen über die Netzwerkschnittstelle
		\item Methoden zur Kommunikation mit dem Interface
		\begin{itemize}
			\item ein/ausschalten der Schnittstelle
			\item Setzen einer eigenen Konfigurationen
		\end{itemize}
		\item Signale zur Kommunikation mit der Applikation
		\begin{itemize}
			\item Zustandsänderungen oder Eigenschaftsänderungen der Schnittstelle
		\end{itemize}
	\end{itemize}
\end{frame}

\subsection{libnutwireless}
\begin{frame}[<+-| alert@+>]{wireless client}
	\begin{itemize}
		\item Abstraktion der Kommunikation mit dem wpa\_supplicant und der WirelessExtension
		\item Überblick über den Hauptteil der Bibliothek:
		\begin{itemize}
			\item Bereitstellung von Informationen zur Signalqualität
			\item Es kann nach Netzwerken gescannt werden
			\item Hinzufügen/Entfernen von Netzwerken (auch aus dem Scan)
			\item Hinzufügen eines Ad-Hoc-Netzwerkes mit 4 Klicks und 2 Texteingaben
			\item Konfiguration eines bereits vorhandenen Netzwerks
			\item Speichern der Konfiguration (sofern erlaubt)
		\end{itemize}
		\item Wenn möglich, automatisches Setzen von benötigten Parametern
	\end{itemize}
\end{frame}
