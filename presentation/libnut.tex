
% TODO
\section{Library}
\subsection{Überblick}
\begin{frame}[<+-| alert@+>]{nut client und wireless client}
	\begin{itemize}
		\item libnutclient : Qt-Library für den Client-Teil der DBus Verbindung zum Server.
		\item Verwaltungsstruktur wie im Server

		\item libnutwireless : Qt-Library für die wpa\_supplicant Verwaltung und Serverkommunikation.
	\end{itemize}
\end{frame}

\begin{frame}[<+-| alert@+>]{nut client und wireless client}
	\begin{center}
		\includegraphics<+->{libnutstructure.pdf}
	\end{center}
\end{frame}

\subsection{libnutwireless}
\begin{frame}[<+-| alert@+>]{Überblick}
	\begin{itemize}
		\item Abstraktion der Kommunikation mit dem wpa\_supplicant und der WirelessExtension
		\item Überblick über den Hauptteil der Bibliothek:
		\begin{itemize}
			\item Bereitstellung von Informationen zur Signalqualität
			\item Es kann nach Netzwerken gescannt werden
			\item Hinzufügen/Entfernen von Netzwerken (auch aus dem Scan)
			\item Konfiguration eines bereits vorhandenen Netzwerks
			\item Speichern der Konfiguration (sofern erlaubt)
		\end{itemize}
		\item Wenn möglich, automatisches Setzen von benötigten Parametern
	\end{itemize}
\end{frame}
