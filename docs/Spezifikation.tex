\documentclass[a4paper,10pt]{scrartcl}
\usepackage[utf8x]{inputenc}
\usepackage[a4paper,left=1.5cm,right=1.5cm,top=2.5cm,bottom=2cm]{geometry}

\begin{document}

\begin{Large}\textbf{Netzwerkmanager - nut (Networkutility) - Spezifikation}\end{Large}


\section*{Fähigkeiten}
\begin{itemize}
\item Netzwerkmanager
\item D-Bus-Interface
\item KDE-Client
\item Unterstützung von WPA (voraussichtlich Konfiguration durch wpa\_supplicant)
\item Unterstützung von zeroconf/dhcp/static ip
\item Erkennung von Hardwarezustandsänderung (Kabel aus/einstecken, Wlankarte anschalten etc.)
\item Unterstützung von mehreren IPs pro Interface
\item Eventgesteuerte Ausführung von Skripten für zusätzliche Aktionen
\end{itemize}


\section*{Komponenten}
\subsection*{nuts - Network UTility Server}
Daemon der die Netzwerkkarten (devices) verwaltet \\
und gegebenenfalls je nach Zustand oder Zustandswechsel entsprechende Aktionen ausführt. \\
Idee:
Für jede Netzwerkkarte (device) existieren Umgebungen ($\geq$ 1 pro Karte) mit bestimmten Eigenschaften. \\
In diesen Umgebungen sind die softwareseitigen Netzwerkschnittstellen (IP-Adressen, hier: interfaces).\\
Die Umgebungen können nach mehreren Kriterien ausgewählt werden:
\begin{itemize}
 \item Andere IP-Adressen (und optional passende MAC-Adresse) im Netzwerk
 \item evtl. weitere Kriterien
\end{itemize}
Standardumgebung für jede Netzwerkkarte ist ``default'' mit IP-Adresse per dhcp. \\
Wenn die Netzwerkkarte nicht aufgeführt ist, so wird sie nicht verwaltet.\\
Die Kommunikation mit dem Server erfolgt über D-Bus.
D-Bus-Interface Dienste:
\begin{itemize}
 \item Umgebungen setzen
 \item Umgebung erfragen
 \item Manuelle Konfiguration der Netzwerkkarte (erfragt vom Server/aufgerufen vom NUTser (Konfig der Interfaces) static user)
\end{itemize}

\subsection*{nut - Network UTility}
Kommandozeilensteuerung für nuts.
\subsection*{knut/qnut}
KDE bzw. Qt Steuerprogramm für nut. Der NUTser (Benutzer) kann damit die Wlan-Verbindungen wechseln, \\
neue aufbauen oder wird vom Server nach einer manuellen Konfiguration gefragt bzw. kann diese selbst verwalten.

\section*{Voraussichtliche Arbeitsteilung}
Oliver: Knut \\
Stefan: Nuts \\
Daniel: D-Bus

\end{document}
